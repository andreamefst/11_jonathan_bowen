\documentclass[a4paper]{article}

\usepackage{color}
\usepackage{url}
\usepackage[utf8]{inputenc}
\usepackage{graphicx}
\usepackage[english,serbian]{babel}
\usepackage[unicode]{hyperref}
\hypersetup{colorlinks,citecolor=green,filecolor=green,linkcolor=blue,urlcolor=blue}

\begin{document}

\title{Džonatan Bouen\\ \small{Seminarksi rad u okviru kursa\\Tehničko i naučno pisanje\\Matematički fakultet}}

\author{Vladimir Jovanović\\vlad.jov.096@gmail.com}
\maketitle

\abstract{\textbf{Džonatan P. Bouen} (rođen 1956. godine) je britanski stručnjak za računare.}

\tableofcontents
\newpage

\section{Pregled}
\label{sec:pregled}
Džonatan Bouen je predstavnik „Museophile Limited” kompanije i profesor emeritus na London South Bank univerzitetu, gde je rukovodio Centrom za primenjene formalne metode. Bio je profesor računarske nauke na Univerzitetu u Birmingemu, gostujući profesor na institutu Prat (Njujork), Univerziteta u Vestminsteru, Kings koledža (London), i gostujući akademik na Londonskom univerzitetskom koledžu.

\section{Obrazovanje}
Rođen je u Okfsordu, sin Hamfrija Bouena, a školovao se u tzv. „Dragon School” (Oksford) i u školi u Brajanstonu pre nego što je maturirao na Univerzitetskom koledžu u Oksfordu, gde je stekao zvanje magistra inžinjerskih nauka. 
\section{Karijera}
Bouen je kasnije radio na koledžu „Imperial College” u Londonu, kompjuterskoj laboratoriji univerziteta u Oksfordu (sada odseku za informatičke tehnologije na univerzitetu u Oksfordu), na Univerzitetu u Redingu i univerzitetu „London South Bank”. Njegov rani rad bio je zasnovan generalno na formalnim metodama, a kasnije naročito na „Z-notaciji” (engl. \textit{the Z notations}). Bio je predstavnik grupe „Z-korisnika” (engl. \textit{the Z user group}) od ranih 1990-ih godina do 2011. godine. Proglašen je predstavnikom britanskog kompjuterskog društva „FACS” (engl. \textit{Specialist Group on Formal Aspects of Computing Science}) 2002. godine. Od 2005. godine, Bouen je pomoćnik glavnog urednika novina „Inovacije u sistemu i softverskom inženjerstvu”. Pored toga, saradnik je i urednika naslovne strane novina „ACM Computing Surveys”, pokrivajući oblast softverskog inženjerstva i formalnih metoda. Od 2008.–2009. godine, bio je saradnik u „Praxis High Integrity Systems” i radio na velikom industrijskom projektu koristeći Z-notacije. 
\section{Odabrane knjige}


\end{document}